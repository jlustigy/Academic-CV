% LaTeX resume using res.cls
\documentclass[margin,10pt]{res}
%\usepackage{helvetica} % uses helvetica postscript font (download helvetica.sty)
%\usepackage{newcent}   % uses new century schoolbook postscript font 

\setlength{\textwidth}{6in} % set width of text portion
\newcommand{\code}[1]{\texttt{#1}}
\usepackage{hyperref}
\usepackage[dvipsnames]{xcolor}
\hypersetup{
    colorlinks=true,       % false: boxed links; true: colored links
    linkcolor=red,          % color of internal links (change box color with linkbordercolor)
    citecolor=green,        % color of links to bibliography
    filecolor=magenta,      % color of file links
    urlcolor=MidnightBlue           % color of external links
}

\usepackage{enumitem}
\setlist[itemize,1]{leftmargin=\dimexpr 26pt-3mm}

\begin{document}

%% Center the name over the entire width of resume:
\moveleft.5\hoffset\centerline{\textsc{\Large Jacob Lustig-Yaeger}}
\moveleft.5\hoffset\centerline{Curriculum Vitae}
%
%% Draw a horizontal line the whole width of resume:
\moveleft\hoffset\vbox{\hrule width 0.93\resumewidth height 1pt}
\vspace{-12pt}
\hspace{-1in}
\begin{minipage}[t]{0.93\resumewidth}
Email: \href{mailto:jlustigy@uw.edu}{jlustigy@uw.edu} \hfill GitHub: \href{https://github.com/jlustigy}{jlustigy}\\
Web: \url{https://jlustigy.github.io/}
\end{minipage}


%% Select character to use as the bullet marker. 
%% Comment out to get normal dark filled circle bullets
\def\labelitemi{---}	

\begin{resume}

\section{Office \\Address}
Department of Astronomy, University of Washington \\
Physics-Astronomy Bldg, B319 \\
Box 351580 \\
Seattle, WA 98195-1580 \\

\section{Education} 
University of Washington, Seattle, WA \hfill 2014 --\\
Graduate student in Astronomy and Astrobiology (dual-title PhD program)  

University of Washington, Seattle, WA \hfill 2014 -- 2016\\
M. Sci. in Astronomy

University of California, Santa Cruz, CA  \hfill 2009 -- 2013\\
B.S.\ with Honors in Physics \\
Minor in Mathematics\\

\section{Research \\Experience} 
                {\sl Graduate Research Assistant}: Virtual Planetary Laboratory   \hfill          Sept 2014 --\\
                Extrasolar planets, their atmospheres, \& habitability with Dr. Victoria Meadows
                    \begin{itemize}  \itemsep -1pt %reduce space between items
                        \item Lead developer of a retrieval model for the analysis of terrestrial exoplanet spectra 
                        \item Experience simulating and analyzing radiative transfer, photochemical, climate, telescope noise, and exoplanet mapping models
                        \item Programming in Python, Julia, IDL, \& Fortran
                    \end{itemize}  

                {\sl Junior Specialist}: University of California, Santa Cruz \hfill          Dec 2013 -- Aug 2014\\
                Hot Jupiter atmospheres with Dr. Jonathan Fortney and Dr. Michael Line
                    \begin{itemize}  \itemsep -1pt %reduce space between items
                        \item Wrote Python code to analyze the emission spectra of exoplanets observed during secondary eclipse
                        \item Gained experience using Bayesian methods of parameter estimation
                    \end{itemize}  
                    
                {\sl Undergraduate Researcher}: University of California, Santa Cruz \hfill          June 2012 -- Dec 2013\\
                Extrasolar planet and brown dwarf atmospheric opacity sources with Jonathan Fortney
                    \begin{itemize}  \itemsep -1pt %reduce space between items
                        \item Wrote IDL code to calculate, tabulate, and plot weighted mean opacities over a wide range of atmospheric temperatures, pressures, and metallicities\\
                    \end{itemize}
   
\section{Teaching \\Experience} 
                {\sl Research Mentor} \hfill Sept 2016 --
                    \begin{itemize}  \itemsep -1pt %reduce space between items
                        \item \href{https://www.washington.edu/undergradresearch/guadalupe-tovar/}{Guadalupe Tovar} (UW Undergraduate)
                    \end{itemize}
                {\sl Teaching Assistant}: Department of Astronomy, University of Washington   \hfill          Sept 2014 -- June 2015\\
                Led two biweekly sections for undergraduate students
                    \begin{itemize}  \itemsep -1pt %reduce space between items
                        \item ASTR 101 (Spring 2015; Autumn 2014)
                        \item ASTR 150 (Winter 2015)
                    \end{itemize}  
                
                {\sl Math \& Writing Tutor}: Learning Support Services, UCSC   \hfill          Sept 2010 -- June 2012\\
                Instructed students in college level mathematics and writing as a group and drop-in tutor\\

                 
                                  
\section{Honors \\ \& Awards}
                 \begin{itemize}  \itemsep -1pt %reduce space between items
                    %\item Astrobiology Fellow at University of Washington Astrobiology Program (2014)
                    \item Honors undergraduate thesis in physics (2013)
                    \item University Honor, \textit{cum laude} at University of California, Santa Cruz (2013)\\
                 \end{itemize}  

\section{Publications} 
                \begin{itemize}  
                    \item Luger, R., \textbf{Lustig-Yaeger, J.}, Fleming, D. P., Tilley, M. A., Agol, E, Meadows, V. S., Deitrick, R., \& Barnes, R. (2017). \href{http://adsabs.harvard.edu/abs/2017ApJ...837...63L}{``The Pale Green Dot: A Method to Characterize Proxima Centauri b using Exo-Aurorae''}. \textit{The Astrophysical Journal}, 837, 63.
                    \item Meadows, V. S., Arney, G. N., Schwieterman, E. W., \textbf{Lustig-Yaeger, J.}, Lincowski, A. P., Robinson, T.,  Domagal-Goldman, S. D., Barnes, R. K., Fleming, D. P., Deitrick, R., Luger, R., Driscoll, P. E., Quinn, T. R., Crisp, D. (2017, in review). \href{http://adsabs.harvard.edu/cgi-bin/bib_query?arXiv:1608.08620}{``The Habitability of Proxima Centauri b II: Environmental States and Observational Discriminants''}. \textit{arXiv preprint arXiv:1608.08620.}
                    \item Barnes, R., Deitrick, R., Luger, R., Driscoll, P. E., Quinn, T. R., Fleming, D. P., Arney, G., Crisp, D., Domagal-Goldman, S. D., Lincowski, A. P., \textbf{Lustig-Yaeger, J.}, \& Schwieterman, E. (2017, in review). \href{http://adsabs.harvard.edu/cgi-bin/bib_query?arXiv:1608.06919}{``The Habitability of Proxima Centauri b I: Evolutionary Scenarios''}. \textit{arXiv preprint arXiv:1608.06919.}
                    \item Greene, T. P., Line, M. R., Montero, C., Fortney, J. J., \textbf{Lustig-Yaeger, J.}, \& Luther, K. (2016). \href{http://adsabs.harvard.edu/abs/2016ApJ...817...17G}{``Characterizing transiting exoplanet atmospheres with JWST''}. \textit{The Astrophysical Journal}, 817(1), 17.
                    \item Freedman, R. S., \textbf{Lustig-Yaeger, J.}, Fortney, J. J., Lupu, R. E., Marley, M. S., \& Lodders, K. (2014). \href{http://adsabs.harvard.edu/abs/2014ApJS..214...25F}{``Gaseous Mean Opacities for Giant Planet and Ultracool Dwarf Atmospheres over a Range of Metallicities and Temperatures''}. \textit{The Astrophysical Journal Supplement Series}, 214(2), 25.\\
                 \end{itemize}
                 
\section{Conference \\Presentations} 
                {\sl Contributed Talks}
                \begin{itemize}
                    \item \textbf{Lustig-Yaeger, J.}, Tovar, G., Fujii, Y., Schwieterman, E., \& Meadows, V.\ (2017). \href{http://www.lpi.usra.edu/meetings/abscicon2017/pdf/3558.pdf}{``Mapping Surfaces and Clouds on Terrestrial Exoplanets Observed with Next-Generation Coronagraph-Equipped Telescopes''}. Astrobiology Science Conference, \#3558
                    \item \textbf{Lustig-Yaeger, J.}, Line, M.~R., \& Fortney, J.~J.\ (2015). \href{http://adsabs.harvard.edu/abs/2015AAS...22512403L}{``On the Confidence of Molecular Detections in the Atmospheres of Exoplanets from Secondary Eclipse Spectra''}. American Astronomical Society Meeting Abstracts, 225, \#124.03
                \end{itemize}
                {\sl Posters}
                \begin{itemize}  
                    \item \textbf{Lustig-Yaeger, J.}, Schwieterman, E., Meadows, V., \& Fujii, Y. (2016). \href{http://adsabs.harvard.edu/abs/2016DPS....4812234L}{``Modeling Earth's Disk-Integrated, Time-Dependent Spectrum: Applications to Directly Imaged Habitable Planets''}. AAS/Division for Planetary Sciences Meeting Abstracts, 48, \#122.34
                    \item \textbf{Lustig-Yaeger, J.}, Meadows, V., Schwieterman, E. W., \& Robinson, T. (2016). \href{http://www.exoplanetscience.org/speakers}{``Modeling Earth’s Disk-Integrated Spectrum through a Lunar Month: 
                    Applications to Directly Imaged Habitable Exoplanets''}. Exoplanets I
                    \item \textbf{Lustig-Yaeger, J.}, Meadows, V., Line, M., \& Crisp, D.\ (2015). \href{http://adsabs.harvard.edu/abs/2015DPS....4741610L}{``A Novel Approach to Atmospheric Retrieval for Small Exoplanets''}. AAS/Division for Planetary Sciences Meeting Abstracts, 47, \#416.10
                    \item \textbf{Lustig-Yaeger, J.}, Line, M., Fortney, J.~J., \& Meadows, V.\ (2015). \href{http://www.hou.usra.edu/meetings/abscicon2015/pdf/7558.pdf}{``Detecting Molecules in Exoplanet Atmospheres: Lessons Learned from Hot Jupiters''}. Astrobiology Science Conference, \#7558
                    \item \textbf{Lustig-Yaeger, J.},  Line, M.~R., \& Fortney, J.~J.\ (2014). \href{http://www2.lowell.edu/workshops/coolstars18/abstracts-posters/poster-abstract-267.html}{``On the Detection Significance of Molecules in Exoplanets from Secondary Eclipse Observations''}. Cool Stars, 18, \#267
                    \item \textbf{Lustig-Yaeger, J.}, Fortney, J.~J., Freedman, R., Marley, M.~S., \& Lupu, R.~E.\ (2014). \href{http://adsabs.harvard.edu/abs/2014AAS...22334704L}{``Gaseous Mean Opacities for Giant Planet and Brown Dwarf Atmospheres''}. American Astronomical Society Meeting Abstracts \#223, \#347.04\\
                 \end{itemize}
                 
 \section{Public Talks}
                \begin{itemize}  
                    \item ``BREAKING: Terrestrial Exoplanet Discovered in the Habitable Zone of Proxima Centauri'' Astronomy on Tap, Peddler Brewing Company, Seattle, WA. August 24, 2016.
                 \end{itemize}

\end{resume}
\end{document}
